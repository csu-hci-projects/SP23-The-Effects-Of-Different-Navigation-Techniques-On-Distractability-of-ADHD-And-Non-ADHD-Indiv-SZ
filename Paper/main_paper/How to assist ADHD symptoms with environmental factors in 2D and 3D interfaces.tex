%%
%% This is file `sample-sigconf.tex',
%% generated with the docstrip utility.
%%
%% The original source files were:
%%
%% samples.dtx  (with options: `sigconf')
%%
%% IMPORTANT NOTICE:
%%
%% For the copyright see the source file.
%%
%% Any modified versions of this file must be renamed
%% with new filenames distinct from sample-sigconf.tex.
%%
%% For distribution of the original source see the terms
%% for copying and modification in the file samples.dtx.
%%
%% This generated file may be distributed as long as the
%% original source files, as listed above, are part of the
%% same distribution. (The sources need not necessarily be
%% in the same archive or directory.)
%%
%% The first command in your LaTeX source must be the \documentclass command.
\documentclass[sigconf]{acmart}
 
\usepackage{graphicx}
 
\usepackage{pgfplots}
\pgfplotsset{compat=1.8}
\usepgfplotslibrary{statistics}
\usetikzlibrary{patterns}
 
 
%%
%% \BibTeX command to typeset BibTeX logo in the docs
\AtBeginDocument{%
  \providecommand\BibTeX{{%
    \normalfont B\kern-0.5em{\scshape i\kern-0.25em b}\kern-0.8em\TeX}}}
 
%% Rights management information.  This information is sent to you
%% when you complete the rights form.  These commands have SAMPLE
%% values in them; it is your responsibility as an author to replace
%% the commands and values with those provided to you when you
%% complete the rights form.

%%
%% Submission ID.
%% Use this when submitting an article to a sponsored event. You'll
%% receive a unique submission ID from the organizers
%% of the event, and this ID should be used as the parameter to this command.
%%\acmSubmissionID{123-A56-BU3}
 
%%
%% The majority of ACM publications use numbered citations and
%% references.  The command \citestyle{authoryear} switches to the
%% "author year" style.
%%
%% If you are preparing content for an event
%% sponsored by ACM SIGGRAPH, you must use the "author year" style of
%% citations and references.
%% Uncommenting
%% the next command will enable that style.
%%\citestyle{acmauthoryear}
 
%%
%% end of the preamble, start of the body of the document source.
\begin{document}
 
%%
%% The "title" command has an optional parameter,
%% allowing the author to define a "short title" to be used in page headers.
\title{How to assist ADHD symptoms with environmental factors in 2D and 3D interfaces}
 
%%
%% The "author" command and its associated commands are used to define
%% the authors and their affiliations.
%% Of note is the shared affiliation of the first two authors, and the
%% "authornote" and "authornotemark" commands
%% used to denote shared contribution to the research.
\author{Ian Dutt}
\affiliation{%
  \institution{Colorado State University}
  \city{Fort Collins}
  \state{Colorado}
  \country{USA}}
\email{Ian.Dutt@colostate.edu}

\author{Zach Shimpa}
\affiliation{%
  \institution{Colorado State University}
  \city{Fort Collins}
  \state{Colorado}
  \country{USA}}
\email{Zach.Shimpa@colostate.edu}


\author{Alizee Barney}
\affiliation{%
  \institution{Colorado State University}
  \city{Fort Collins}
  \state{Colorado}
  \country{USA}}
\email{Alizee.Barney@colostate.edu}
 

 
 
%%
%% By default, the full list of authors will be used in the page
%% headers. Often, this list is too long, and will overlap
%% other information printed in the page headers. This command allows
%% the author to define a more concise list
%% of authors' names for this purpose.
\renewcommand{\shortauthors}{}
 
%%
%% The abstract is a short summary of the work to be presented in the
%% article.
\begin{abstract}
this is the abstract section...
\end{abstract}
 
%%
%% The code below is generated by the tool at http://dl.acm.org/ccs.cfm.
%% Please copy and paste the code instead of the example below.
%%
\begin{CCSXML}
\begin{CCSXML}
<ccs2023>
<concept>
<concept_id>10002951.10003317.10003347.10003350</concept_id>
<concept_desc>How to assist ADHD symptoms with environmental factors in 2D and 3D interfaces</concept_desc>
<concept_significance>500</concept_significance>
</concept>
</ccs2023>
\end{CCSXML}
 
%% 
\ccsdesc[500]{Put keywords here}
 
%%
%% Keywords. The author(s) should pick words that accurately describe
%% the work being presented. Separate the keywords with commas.
\keywords{}
 
%%
%% This command processes the author and affiliation and title
%% information and builds the first part of the formatted document.
\settopmatter{printacmref=false} %% Removes some of the items that the template adds on
\maketitle
 
\section{Background}
This is where the Background research goes, all the past paper on this topic
 
\section{Introduction}
 We introduce our research project here.

\section{References}
 
Examples:

[1] J. Ben Schafer, Joseph Konstan, and John Riedl. 1999. Recommender systems in e-commerce. In Proceedings of the 1st ACM conference on Electronic commerce (EC '99). Association for Computing Machinery,
New York, NY, USA, 158–166.\\
https://doi.org/10.1145/336992.337035
 
\end{document}
\endinput
%%
%% End of file `sample-sigconf.tex'.
